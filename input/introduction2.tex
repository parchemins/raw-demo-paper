
\section{Introduction}

SPARQL queries, such as the query $Q_{604}$ of Figure~\ref{fig:raw_screenshot} cannot be executed on public SPARQL
endpoints due to their fair-use policy i.e., they are stopped after the 60s
returning only partial results. Because of this limitation, many
use-cases are such as building summaries or computing statistics are
impossible using only public SPARQL endpoints. This is a major issue,
especially for building a decentralized eco-system of public SPARQL
endpoints.

Sampling-based Approximate Query Processing
(S-AQP)~\cite{DBLP:conf/sigmod/AgarwalMKTJMMS14} may tackle this issue
for some use-cases such as computing large-scale
statistics~\cite{soulet2019anytime,10.1007/978-3-319-18818-8_14},
knowledge graph embeddings~\cite{ristoski2016rdf2vec}, join orders
optimisation~\cite{DBLP:conf/cidr/LeisRGK017}, approximate
aggregations~\cite{wang2022approximate},
summaries~\cite{10.1007/978-3-030-49461-2_10}, and exploratory
queries~\cite{DBLP:conf/sigmod/AgarwalMKTJMMS14}. By evaluating SPARQL
queries only a sample of data, S-AQP drastically reduce execution time
returning approximated results with error estimation.  The challenge
for S-AQP support in SPARQL is to garantee that the evaluation of
SPARQL query under sampling never time-out on public SPARQL endpoints.

In this demonstration, we extended JENA to efficiently support S-AQP
for conjunctive SPARQL queries. \NAME evaluates a query $Q$ over
dataset $D$ by drawing  random walks guided by $Q$ over $D$ as defined in
\WANDER~\cite{li2019wanderjoin}. We rely on random walks for the
following reasons:
%
\begin{inparaenum}[(i)]
%
\item As each random walk is independant, a
user can draw multiple random walks distributed in time, respecting
the fair policy of public SPARQL endpoint and getting random results
in a pay-as-go approach.
%
\item As the complexity of drawing one random walk is logarithmic, it is
possible to draw thousands of random walk per second. Thanks to
traditional BTree indexes, which are widely used to index RDF data,
\NAME computes a random walk in $\mathcal{O}(|Q|log(|D|))$ where $|Q|$
is the number of triple pattern in $Q$ and $|D|$ is the number of
triples in $D$.
%
\item As random walks produce random partial results, estimating the
cardinality of complete allows users estimate the representativeness
of the sample. For example, for the query $Q_{604}$, after 35s of
processing, it is possible to find 1534 results, but users kwow that
the query returns approximately 25M results. Attaching probalities to random path allows to estimate
the cardinality of queries\cite{li2019wanderjoin} with high accuracy
as descibed in G-CARE benchmark~\cite{DBLP:conf/sigmod/ParkKBKHH20}
\end{inparaenum}


%
% Note sure about the following sentences... Do we want to promote the pay-as-you-go evaluation strategy?
%
%One advantage of random walks is to provide a pay-as-you-go service. % for all use-cases.
%The data collected by sampling $Q$ over $D$ can be merged with previously collected
%data to get more results and a better estimate of the total number of results. With an
%unlimited budget, \NAME eventually returns all the results of $Q$.

For the demonstration, we load the 1,2 billion triples Wikidata dataset
provided by WDBench~\cite{angles2022wdbench}, and
we re-execute the 16 queries that time out on Jena using \NAME. %S-AQP as explained in Section~\ref{proposal}. 
%
% Hala
%  The S-AQP results shed light on the reasons for these queries time out and provide users with valuable  information about
% Thanks to S-AQP results, users can understand
% why these queries time out. Overall, this demonstration highlights
\NAME aims to highlight
\begin{inparaenum}[(i)]
\item the benefits of S-AQP for an ecosystem of public SPARQL endpoints,
\item the feasibility of implementing S-AQP on well-known engines.
\end{inparaenum}

% - There's nothing about fair use policies?
% - There's just a little about how to guarantee logarithmic time complexity for triple pattern sampling.
%   In the conclusion it's written that "\NAME demonstrates how S-AQP can be easily integrated into a
%   representative SPARQL endpoint" but this part is not explained in the paper...
% - What about the SAMPLE keyword of SPARQL?

%%% Local Variables:
%%% mode: latex
%%% TeX-master: "../paper.tex"
%%% End:
