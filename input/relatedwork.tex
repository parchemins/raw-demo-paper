

\section{Related Work}

In the SQL world, sampling is part of SQL standard since 2003 with the
TABLESAMPLE clause in SELECT
queries\foonote{\URL{https://www.postgresql.org/docs/current/sql-select.html}}. TABLESAMPLE
after a table name in a select query specifies that sampling have to
be used when retreiving rows in that table. Such approach does not
allow to estimate the total number of results as proposed in
WanderJoin\cite{DBLP:journals/tods/LiWYZ19}.  In SPARQL, it is possible to draw random
results from a SPARQL query using \verb+ ORDER BY RAND() LIMIT 100'+
or using random \verb+OFFSET+. However, such queries are not garanteed
to terminate in fixed time, making sampling of query results at least as
difficult as evaluating a query. Virtuoso allows to draw
random triples more
efficiently~\footnote{https://docs.openlinksw.com/virtuoso/rndsalltr/}
with a dedicated user defined function, but it is restricted to triple
pattern and there is no execution time garantee. Stardog allows also
random triple retreival
~\footnote{\url{https://docs.stardog.com/query-stardog/sampling-service#sampling-service.}},
but it is still restricted to a triple pattern, the complexity of
random access is not detailed and there is no statistics about results
as in SQL.

Internally, sampling is commonly used in triplestore engines mainly to
estimate cardinalities of SPARQL operators and optimizing
queries~\cite{DBLP:conf/cidr/LeisRGK017} including property path evaluation\cite{10.1007/978-3-031-33455-9_3}. In this paper, we propose to
make it available for end-user by exposing estimations with random
results. Externally it has been used for large scale anaytics of
linked data~\cite{soulet2019anytime}, and aggregate
estimation\cite{li2016wanderjoin}, embedding
generation~\cite{ristoski2016rdf2vec}, summaries~\cite{10.1007/978-3-030-49461-2_10}.

Debattista et al. [4] propose approximating specific quality metrics
for large, evolving datasets based on
samples~\cite{10.1007/978-3-319-18818-8_14}

Rietveld et al. [16] aim to obtain samples that entail as many of the
original answers to typical SPARQL queries~\cite{10.1007/978-3-319-11915-1_6}



%%% Local Variables:
%%% mode: latex
%%% TeX-master: "../paper.tex"
%%% End:
