
\section{Introduction (from sampling)}

Sampling-based approximate query processing (S-AQP)
~\cite{DBLP:conf/sigmod/AgarwalMKTJMMS14} has many use-case in RDF
including computing large scale
statistics\cite{soulet2019anytime,10.1007/978-3-319-18818-8_14},
computing embeddings with random walks\cite{ristoski2016rdf2vec}, join
orderings\cite{DBLP:conf/cidr/LeisRGK017}, approximated
aggregations\cite{DBLP:journals/tods/LiWYZ19}, summaries
computation~\cite{10.1007/978-3-030-49461-2_10}. However, current
SPARQL endpoints are not able to efficiently deliver for a given query
results drawn uniformly at random. To illustrate, drawing a triple
uniformly at random can be achieved by executing:
%
\verb+select * {?s ?p ?o} LIMIT 1 OFFSET r+
%
where $r$ is a random number between 0 and the cardinality of the
graph~\cite{soulet2019anytime}. However, if r>100M, all query time-out
on Wikidata returning no result\footnote{other methods relying on
  ORDER BY rand() is also not ensured to terminate in bounded time
}. Of course, it is possible to draw a random triple in logarithmic
time using the internal index of a TripleStore, but this is not
accessible for the end-user. In this demo, we demonstrate how we
extended JENA to support efficiently sampling-based approximate
processing. Given a SPARQL query and a budget in time, JENA-RAW is
able to deliver random results along with a cardinality estimation of
the total number of results.






\section{Introduction (from quota)}

Public SPARQL endpoints as Wikidata or DBPedia allow anyone to execute
any SPARQL queries. However, due to fair use policies of public SPARQL
endpoints, there is no guarantee of termination with complete
results. On Wikidata, queries are stopped after 60s returning partial
results~\footnote{\url{https://www.mediawiki.org/wiki/Wikidata_Query_Service/User_Manual}},
On Dbpedia, the maximum execution time is set to 120 second with a
maximum 10000
results~\footnote{\url{https://www.dbpedia.org/resources/sparql/#ratesandlimits}}. Consequently,
it exist a class of SPARQL queries that time-out when executed on
public SPARQL endpoint~\cite{DBLP:conf/semweb/MalyshevKGGB18} that is
the base for the Wikidata benchmark\cite{angles2022wdbench}.

It is impossible for a SPARQL endpoint to ensure that any query
returns complete results in fixed time. However, it is possible to
ensure to return a sample of results in fixed time, along with an
estimation of the cardinalities of complete results. For example, the
$Q_1$ of figure~\ref{fig:q1-nojo} timeout on Wikidata, returning
partial non-random results. However, it is possible to sample the
evaluation of Q1, returning random results along with an estimation of
the cardinality of results with a confidence interval. For a budget of
1s, on JENA-RAW, we obtained 50 results on potentially 1000 results
more or less 50.

Sampled results can be incrementally improved following a
pay-as-go approach ie. resending the same
query for another budget of 1s allows to get other random results with
possible duplicates, but with an higher accuracy on cardinality.



\begin{figure}[t]
 \begin{center}
  \subfloat [Query $Q_1^{J_1}$ time-out >60s] {\label{fig:q1-nojo}
   \adjustbox{valign=T}{
    \resizebox{0.45\textwidth}{!}{
     \lstinputlisting[basicstyle=\scriptsize\sffamily,
     language=sparql,numbers=none,columns=fixed,
     showstringspaces=false]{./figures/q1w.rq}}}}
  \subfloat [Query $Q_1^{J_2}$ forced join order $\sim 451ms$] {\label{fig:q1j2}
   \adjustbox{valign=T}{
    \resizebox{0.45\textwidth}{!}{
     \lstinputlisting[basicstyle=\scriptsize\sffamily,
     language=sparql,numbers=none,columns=fixed,
     showstringspaces=false]{./figures/q1w-jo.rq}}}}
 \end{center}
 \caption{The query $Q_1$ searches for creative works and the list of
  fiction works that inspired them. $Q_1$ time-out on
  the Wikidata online server (>60s)}
 \label{fig:q1}
\end{figure}

Sampling query evaluation has many practical use-cases that is
currently limited as computing
large scale statistics\cite{soulet2019anytime}, embeddings with random
walks\cite{ristoski2016rdf2vec}, join
orderings\cite{DBLP:conf/cidr/LeisRGK017}, approximated
aggregations\cite{DBLP:journals/tods/LiWYZ19}.




To validate our
approach, we extended JENA with a sampling interface and show what can
be obtained with fixed time for queries that traditionaly time-out on
public endpoints.




Random walks are important but do not have efficient implementations,
or unsufficient API. More importantly, they are precluded to servers
and they do not let outsiders use them.

Could go for triple/quad patterns and it would be enough. But the more
we push to server the more efficient.

Especially relevant for SPARQL since we often have all indexes.

We limit ourself to a subset of SPARQL for now. We do not process
property paths although it would be possible. Optionals, set minus, are
difficult.

\todo{MUST shoot a short video to showcase \NAME. And link it.}

Random walks enable few use-cases (from most easy to most demanding)
such as:
\begin{asparadesc}
\item [Summaries] needs random walk values.
\item [pyRDF2Vec~\cite{steenwinckel2023pyrdf2vec}] needs random walk values.
\item [FedUP] needs the graphs of random walks.
\item [Join ordering~\REF] needs cardinalities of random walks.
\item [Sparklis~\cite{ferre2017sparklis}] needs random walks (at
  least) and cardinality (optional).
\item [Wander Join~\cite{li2016wanderjoin}] needs the cardinality of
  random walks, failed and succeeded.
\end{asparadesc}

%%% Local Variables:
%%% mode: latex
%%% TeX-master: "../paper.tex"
%%% End:
