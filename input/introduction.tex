
\section{Introduction}

Public SPARQL endpoints cannot fully execute many SPARQL queries due
to their quotas and fair-use policies. After 60 seconds, they stop
their execution and return only partial results, forbidding many use
cases such as building summaries or computing large-scale statistics.
This constitutes a major issue for building decentralized ecosystems
of public SPARQL endpoints.

Sampling-based Approximate Query Processing
(S-AQP)~\cite{DBLP:conf/sigmod/AgarwalMKTJMMS14} tackles this issue
for use cases such as computing large-scale
statistics~\cite{soulet2019anytime,10.1007/978-3-319-18818-8_14},
knowledge graph embeddings~\cite{ristoski2016rdf2vec}, join
orders~\cite{DBLP:conf/cidr/LeisRGK017}, approximate
aggregations~\cite{wang2022approximate},
summaries~\cite{10.1007/978-3-030-49461-2_10}, and exploratory
queries~\cite{DBLP:conf/sigmod/AgarwalMKTJMMS14}.  By confining query
execution to samples of large datasets, S-AQP drastically reduces
execution time, delivering approximate results with error estimates.
%
To support S-AQP in SPARQL, the evaluation of such queries \emph{must}
both
\begin{inparaenum}[(i)]
\item return random samples, and
\item comply with fair-use policies of public SPARQL endpoints.
\end{inparaenum}
Ad-hoc methods already exist to sample knowledge graphs.  For example, a user could draw random triples from
Wikidata~\cite{soulet2019anytime} by repeatedly executing
\verb|SELECT * {?s ?p ?o} OFFSET r LIMIT 1|, where $r$ is a random
number between $0$ and the dataset size ($0<r<12B$). However, all
queries time out when $r$ is above $100M$.
%Hala: random sampling from the set of results matched by a particular SPARQL triple pattern
Some triple stores propose home-made methods for sampling triple
patterns%
\footnote{\url{https://docs.stardog.com/query-stardog/sampling-service\#sampling-service}}\textsuperscript{,}\footnote{\url{https://docs.openlinksw.com/virtuoso/rndsalltr/}},
but these solutions are limited to single triple patterns and the
underlying complexity is not established.

In this demonstration, we introduce \NAME, an open source extension of
Apache Jena that efficiently supports S-AQP for conjunctive SPARQL
queries. \NAME evaluates a query $Q$ over a dataset $D$ by drawing
random walks guided by $Q$, as defined in
\WANDER~\cite{li2019wanderjoin}. Random walks have two desirable
properties:
\begin{inparaenum}[(i)]
\item As each random walk is \emph{independent}, a user can distribute
  its query execution into multiple requests, hence collecting and
  merging random walks in a pay-as-you-go fashion without impairing
  the fair-use policy of public SPARQL endpoints.
\item As each random walk enjoys a \emph{logarithmic time complexity},
  endpoints can draw thousands of random walks per second: the widely
  used BTree indexes of RDF stores allow endpoints to compute each
  random walk in $\mathcal{O}(|Q| \log |D|)$ where $|Q|$ is the number
  of triple/quad patterns in $Q$ and $|D|$ is the number of
  triples/quad in $D$.
\end{inparaenum}

\NAME produces random partial results instead of complete
results. Without cardinality estimates, users cannot appreciate the
quality of provided samples. In Figure~\ref{fig:raw_screenshot}, \NAME
returns $1434$ random results after $35$ seconds. By adding an
estimate of the total number of results ($26 \pm 3M$), the user
acknowledges that her sample represents but a tiny fraction of the
whole results.  Provided estimates prove highly accurate as long as
random walks remain paired with their probability of being
drawn~\cite{DBLP:conf/sigmod/ParkKBKHH20}. Despite starting from $35
\pm 15M$, \NAME improved over time, reaching $26\pm 3M$ while the
actual number of results is $25M$.

For the demonstration, we load the dump of Wikidata provided by
WDBench~\cite{angles2022wdbench} that comprises $1.2B$ triples.  We
let users choose a query among the $16$ that time out on Apache
Jena. We execute it using \NAME to highlight
\begin{inparaenum}[(i)]
\item the benefits of S-AQP for ecosystems of public SPARQL endpoints
  and
\item the feasibility of implementing S-AQP on well-known engines.
\end{inparaenum}

%% the $1.2B$ triples Wikidata dataset
%% provided by WDBench~\cite{angles2022wdbench}, and we re-execute the 16
%% queries that time out on Jena using \NAME. %S-AQP as explained in
%% Section~\ref{proposal}.
%
% Hala
%  The S-AQP results shed light on the reasons for these queries time out and provide users with valuable  information about
% Thanks to S-AQP results, users can understand
% why these queries time out. Overall, this demonstration highlights


%%% Local Variables:
%%% mode: latex
%%% TeX-master: "../paper.tex"
%%% End:
