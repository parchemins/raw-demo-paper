
\section{Introduction}

Sampling-based Approximate Query Processing (S-AQP) drastically
reduces query execution time by confining query executions to samples
of large datasets, delivering approximate results with error
estimates~\cite{DBLP:conf/sigmod/AgarwalMKTJMMS14}. S-AQP has many
important use cases in RDF, including computing large-scale
statistics~\cite{soulet2019anytime,10.1007/978-3-319-18818-8_14},
knowledge graph embeddings~\cite{ristoski2016rdf2vec}, better join
orders~\cite{DBLP:conf/cidr/LeisRGK017}, approximate
aggregations~\cite{wang2022approximate},
summaries~\cite{10.1007/978-3-030-49461-2_10}, and exploratory
queries~\cite{DBLP:conf/sigmod/AgarwalMKTJMMS14}.
%



However, since SPARQL~1.1 does not support S-AQP, queries have to be
executed on whole datasets instead of samples to implement these use
cases; and many such queries time out on \emph{public} SPARQL
endpoints due to quota and fair use policy.

Building a robust decentralized ecosystem of SPARQL endpoints remains
challenging.

%leaving end-users with no choice but to materialize and then sample
%results. While sampling could be a way to making online knowledge
%graphs accessible, sampling has the same cost as evaluating a SPARQL
%query, and consequently no guarantee of completion on online SPARQL endpoints.
% In other words, sampling has the same cost as evaluating a
% SPARQL query, the latter having no guarantee of completion on public
% SPARQL endpoints due to fair-use policies. 
% Consequently, sampling has at least the same cost as evaluating
% SPARQL queries, which time out on public SPARQL endpoints.
%
Ad-hoc methods already exist to sample knowledge graphs.  For example,
a user could draw random triples from
Wikidata~\cite{soulet2019anytime} by repeatedly executing
\verb|SELECT * {?s ?p ?o} OFFSET r LIMIT 1|, where $r$ is a random
number between $0$ and the dataset size ($0<r<12B$). However, all
queries time out when $r$ is above $100M$.
%Hala: random sampling from the set of results matched by a particular SPARQL triple pattern
Some triple stores propose ad-hoc methods for sampling triple
patterns%
\footnote{\url{https://docs.stardog.com/query-stardog/sampling-service\#sampling-service}}\textsuperscript{,}\footnote{\url{https://docs.openlinksw.com/virtuoso/rndsalltr/}},
but these solutions are limited to single triple patterns and the
underlying complexity is not established.
\TODO{Missing a sentence here.}
%
%The challenge for S-AQP support in SPARQL is be sure to deliver random
%results in the current quota of public SPARQL endpoint.

In this demonstration, we introduce \NAME, an extension of Apache Jena
that efficiently supports S-AQP for conjunctive SPARQL queries. \NAME
evaluates a query $Q$ over a dataset $D$ by drawing random walks
guided by $Q$ over $D$ as defined in
\WANDER~\cite{li2019wanderjoin}. Random walks enjoy a few desirable
properties:
\begin{inparaenum}[(i)]
\item As each random walk is \emph{independent}, users can draw
  multiple random walks distributed in time, respecting the fair
  policy of public SPARQL endpoint and getting random results in a
  pay-as-go approach.
\item As the \emph{time complexity} of drawing a random walk is
  logrithmic, it is possible to draw thousands of random walk per
  second. Thanks to traditional BTree indexes, which are widely used
  to index RDF data, \NAME computes a random walk in
  $\mathcal{O}(|Q|log(|D|))$ where $|Q|$ is the number of triple
  pattern in $Q$ and $|D|$ is the number of triples in $D$.
\item As random walks produce random partial results, estimating the
  cardinality of complete allows users estimate the representativeness
  of the sample. Attaching probalities to random path allows to
  estimate the cardinality of
  queries\cite{li2019wanderjoin}. According to
  ~\cite{DBLP:conf/sigmod/ParkKBKHH20}, \WANDER provides the best
  cardinality estimation for online sampling methods.
\end{inparaenum}


%
% Note sure about the following sentences... Do we want to promote the pay-as-you-go evaluation strategy?
%
%One advantage of random walks is to provide a pay-as-you-go service. % for all use-cases.
%The data collected by sampling $Q$ over $D$ can be merged with previously collected
%data to get more results and a better estimate of the total number of results. With an
%unlimited budget, \NAME eventually returns all the results of $Q$.

For the demonstration, we load the 1,2 billion triples Wikidata dataset
provided by WDBench~\cite{angles2022wdbench}, and
we re-execute the 16 queries that time out on Jena using \NAME. %S-AQP as explained in Section~\ref{proposal}. 
%
% Hala
%  The S-AQP results shed light on the reasons for these queries time out and provide users with valuable  information about
% Thanks to S-AQP results, users can understand
% why these queries time out. Overall, this demonstration highlights
\NAME aims to highlight
\begin{inparaenum}[(i)]
\item the benefits of S-AQP for an ecosystem of public SPARQL endpoints,
\item the feasibility of implementing S-AQP on well-known engines.
\end{inparaenum}

% - There's nothing about fair use policies?
% - There's just a little about how to guarantee logarithmic time complexity for triple pattern sampling.
%   In the conclusion it's written that "\NAME demonstrates how S-AQP can be easily integrated into a
%   representative SPARQL endpoint" but this part is not explained in the paper...
% - What about the SAMPLE keyword of SPARQL?

%%% Local Variables:
%%% mode: latex
%%% TeX-master: "../paper.tex"
%%% End:
