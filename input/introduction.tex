
\section{Introduction}

Sampling-based Approximate Query Arocessing (S-AQP) drastically reduces query
execution time by delivering approximated results with error
estimates~\cite{DBLP:conf/sigmod/AgarwalMKTJMMS14}. S-AQP has many important use-cases in RDF including
computing large scale
statistics~\cite{soulet2019anytime,10.1007/978-3-319-18818-8_14},
embeddings with random walks~\cite{ristoski2016rdf2vec}, join
orderings for query optimization~\cite{DBLP:conf/cidr/LeisRGK017},
approximate
aggregations~\cite{wang2022approximate},
summaries~\cite{10.1007/978-3-030-49461-2_10}, exploratory
queries~\cite{DBLP:conf/sigmod/AgarwalMKTJMMS14}.
%
However, current SPARQL endpoints have no native support for S-AQP,
pushing end-users to materialize then sample
results. Consequenly, sampling has at least  the same cost than evaluating
SPARQL queries and many queries just time out on public SPARQL
endpoints.
%
One could draw random triples from Wikidata~\cite{soulet2019anytime}
by repeatedly executing \lstinline[language=SPARQL]|SELECT * {?s ?p  ?o} LIMIT 1 OFFSET r|,
where $r$ is a random number between $0$ and
the dataset size ($0<r<12B$). However, this method times out for $r$
above $100M$. Some triple stores proposes ad-hoc methods for sampling
triple
patterns\footnote{\url{https://docs.stardog.com/query-stardog/sampling-service\#sampling-service.}}
\footnote{\url{https://docs.openlinksw.com/virtuoso/rndsalltr/}}.
However, the sampling is limited to single triple patterns and the
underlying complexity is not established.  On SPARQL, there is nothing
equivalent to the \texttt{TABLESAMPLE} clause that is part of the SQL
standard since 2003. This is a major issue, especially when trying to
build an eco-system based on SPARQL endpoints. With no S-AQP support
on SPARQL endpoints, building summaries directly from SPARQL endpoints
is difficult, as computing VOID descriptions, computing embedding,
optimising join orders for federated queries etc.

% From the state of art~\cite{DBLP:conf/cidr/LeisRGK017}, we know that
% is possible to draw a random triple from a triple pattern in
% logarithmic time using the internal index of a TripleStore. Extending
% to BGP support just multiply this complexity by the number of triple
% pattern in the query. Sampling queries makes sense only if the
% approximation error can also be estimated ie. for a query

  In this demonstration, we extended JENA to efficiently support
  S-AQP for conjunctive core
  SPARQL queries. Given a SPARQL query $Q$ and a budget in time,
  \NAME~\footnote{\url{https://github.com/Chat-Wane/raw-jena}} is able
  to deliver random results along with an estimate of the
  total number of results. Our approach relies on random
 walks as defined in WanderJoin~\cite{li2019wanderjoin}. This approach
 has proven to be accurate on the G-Care
 benchmark~\cite{DBLP:conf/sigmod/ParkKBKHH20} for RDF datasets. Thanks
 to traditional SPO, POS, OSP indexes of RDF, random
 walks for conjunctive queries are executed in $k.log(|G|)$ where
 $k$ is the number of triple patterns in the query and $|G|$ the size
 of the dataset. Thanks to the random walk approach, \NAME converges
 to complete results, i.e. sampling the query multiple times eventually
 returns all results, and the estimated cardinality eventually converges
 to the exact cardinality.

 \noindent For the demonstration, we load the Wikidata
 Benchmark~\cite{angles2022wdbench} of 1,2B triples in \NAME and
 re-execute queries that time-out with S-AQP as described in
 section~\ref{proposal}. Thanks to S-AQP results, users can understand
 why these queries time-out. Overall, this demonstration highlights
\begin{inparaenum}[(i)]
\item the benefits of S-AQP for the public SPARQL endpoints ecosystem,
\item the feasibility of implementation on well-known engines,
\end{inparaenum}

% - There's nothing about fair use policies?
% - There's just a little about how to guarantee logarithmic time complexity for triple pattern sampling.
%   In the conclusion it's written that "\NAME demonstrates how S-AQP can be easily integrated into a
%   representative SPARQL endpoint" but this part is not explained in the paper...
% - What about the SAMPLE keyword of SPARQL?

%%% Local Variables:
%%% mode: latex
%%% TeX-master: "../paper.tex"
%%% End:
