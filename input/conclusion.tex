
\section{Conclusion}

S-AQP has many important use cases that are crucial to build and
maintain ecosystems of public SPARQL endpoints.
%
\NAME demonstrates the feasability of implementing S-AQP on a
representative SPARQL endpoint: Apache Jena.  As each random walk is
independant and enjoys a logarithmic upper bound on its time
complexity, users can collect and merge samples across multiple
executions of the same query. Such a pay-as-you-go approach perfectly
fits the fair-use policies of public SPARQL endpoints.

\noindent In future works, we plan to support full SPARQL queries and
to show how the SPARQL standard could evolve to integrate sampling as
SQL did with the \verb|TABLESAMPLE| clause.
