\documentclass[25pt, a0paper, portrait]{tikzposter}

\usepackage{hyperref}

\RequirePackage[T1]{fontenc}
\RequirePackage{fontspec}
\fontspec[AutoFakeBold=2.5]{IMFellEnglish-Regular.ttf}

\setmainfont{IMFellEnglish}[
  Path=./,
  Extension = .ttf,
  UprightFont = *-Regular,
  ItalicFont = *-Italic,
  BoldFont= *-Regular,
  BoldFeatures={FakeBold=3},
]

\definecolorpalette{sampleColorPalette} {
  \definecolor{colorOne}{named}{green}
  \definecolor{colorTwo}{named}{black}
  \definecolor{colorThree}{named}{cyan}
} {
  \colorlet{backgroundcolor}{colorOne}
}

\definelayouttheme{sample}{
  \usecolorstyle[colorPalette=sampleColorPalette]{sampleColorStyle}
}

\usetheme{sample}




\title{RAW-JENA: Approximate Query Processing for SPARQL Endpoints}

\author{Julien Aimonier-Davat, Minh-Hoang Dang, Pascal Molli, Brice Nédelec, and Hala Skaf-Molli}
\date{November 6--10}
\institute{Nantes Université, CNRS, LS2N}


\begin{document}
\maketitle

\begin{columns}
  \column{0.5}
  \block{S-AQP}{
    And why it is useful
  }
  \column{0.5}
  \block{Ad-hoc sampling}{
    Not great
  }
\end{columns}


\begin{columns} 
  \column{0.5}
  \block{Random walks}{WanderJoin}
  \column{0.5}
  \block{In Apache Jena}{
    Modifying balanced trees\\
    Implementing cardinality estimation
  }
\end{columns}


\begin{columns} % See Section 4.4
  \column{0.33} % See Section 4.4
  \block{Conclusion}{
    Rather simple
  }
  \column{0.33}
  \block{Perspective}{
    More operators
  }
  \column{0.33}
  \block{}{
    \includegraphics[width=0.28\textwidth]{images/qr-code.png}
    \begin{center}
      \url{https://github.com/gdd-nantes/raw-jena}
    \end{center}
  }
\end{columns}

\end{document}
