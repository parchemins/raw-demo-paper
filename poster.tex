\documentclass[25pt, a0paper, portrait]{tikzposter}
\tikzposterlatexaffectionproofoff %% remove the mark on the bottom right

\usepackage{hyperref}
\usepackage{booktabs}
\usepackage{adjustbox}
\usepackage{multicol}

\RequirePackage[T1]{fontenc}
\RequirePackage{fontspec}
\fontspec[AutoFakeBold=3.5]{IMFellEnglish-Regular.ttf}

%% Font of the text, might use a different one for the title
%% that matches better the New York Times front page.
\setmainfont{IMFellEnglish}[
  Path=./,
  Extension = .ttf,
  UprightFont = *-Regular,
  ItalicFont = *-Italic,
  BoldFont= *-Regular,
  BoldFeatures={FakeBold=3.5},
]

%% Custom background and frame, we only need it to be white
\definebackgroundstyle{samplebackgroundstyle}{
  \draw[inner sep=0pt, line width=0pt, color=white, fill=white]
  (bottomleft) rectangle (topright);
}

%% Custom title, nothing really, just remove the ugly frame
\definetitlestyle{sampletitlestyle}{
  width=500mm, roundedcorners=20, linewidth=2pt, innersep=5pt,
  titletotopverticalspace=15mm, titletoblockverticalspace=30mm
}{}

%% Custom block style, we want a title and that's it.
\defineblockstyle{sampleblockstyle}{
    titlewidthscale=1, bodywidthscale=1, titleleft,
    titleoffsetx=0pt, titleoffsety=0pt, bodyoffsetx=0pt, bodyoffsety=0pt,
    bodyverticalshift=0pt, roundedcorners=0,
    titleinnersep=0.5cm, bodyinnersep=0.5cm
}{ %% nothing really
}

%% Colors
\definecolorstyle{samplecolorstyle} {
  \definecolor{colorOne}{named}{yellow}
  \definecolor{colorTwo}{named}{black}
  \definecolor{colorThree}{named}{cyan}
} {
  \colorlet{backgroundcolor}{colorOne}
  \colorlet{framecolor}{black}
  \colorlet{blocktitlefgcolor}{black}
  \colorlet{blocktitlebgcolor}{white}
}

\definelayouttheme{sample}{
  \usecolorstyle{samplecolorstyle}
  \usetitlestyle{sampletitlestyle}
  \usebackgroundstyle{samplebackgroundstyle}
  \useblockstyle{sampleblockstyle}
  \useinnerblockstyle{sampleblockstyle}
}

\usetheme{sample}


\makeatletter
\renewcommand\Huge{\@setfontsize\Huge{135}{135}}
\renewcommand\huge{\@setfontsize\Huge{120}{120}}
\renewcommand\LARGE{\@setfontsize\Huge{70}{70}}
\renewcommand\Large{\@setfontsize\Huge{60}{60}}
\renewcommand\large{\@setfontsize\Huge{50}{50}}
\renewcommand\normalsize{\@setfontsize\Huge{40}{40}} %% between 24-36
\renewcommand\small{\@setfontsize\Huge{20}{20}}
\renewcommand\footnotesize{\@setfontsize\Huge{15}{15}}
\renewcommand\scriptsize{\@setfontsize\Huge{12}{12}}
\renewcommand\tiny{\@setfontsize\Huge{10}{10}} 
\makeatother

\newcommand{\TOPBLOCK}{0.17}

\settitle{
  \centering
  \begin{tabular}{p{\TOPBLOCK\columnwidth}p{0.62\columnwidth}p{\TOPBLOCK\columnwidth}}
    \adjustbox{max width=\TOPBLOCK\columnwidth}{\fbox{\parbox{\dimexpr\linewidth-2\fboxsep-2\fboxrule}
        {``\@author ~ from \textit{\@institute}''}}} &
    \centering\bfseries\Huge\@title &
    \adjustbox{max width=\TOPBLOCK\columnwidth}{{\parbox{\dimexpr\linewidth-2\fboxsep-2\fboxrule}{
          \begin{center}
            \textbf{ISWC edition}
          \end{center}
          
          The 22$^{nd}$ International Semantic Web Conference: The
          premier international forum for the Semantic Web and Linked
          Data Community.}}} \vspace{0.25em}\\ \midrule
    
    & \centering\textsc{\uppercase{Athens, Greece, 6-10 November 2023}} & \hfill\$0.00\\ \bottomrule
  \end{tabular}
}

\newcommand{\HRULE}{\centerline{\rule{0.6667\linewidth}{0.1cm}}}



\title{RAW-JENA}

\author{Julien Aimonier-Davat, Minh-Hoang Dang, Pascal Molli, Brice Nédelec, and Hala Skaf-Molli}
\date{November 6--10}
\institute{Nantes Université, CNRS, LS2N}

\setlength{\columnsep}{1.5cm}

\begin{document}
\maketitle[width=\columnwidth]

\begin{columns}
  \column{0.74}
  \block{Approximate Query Processing for SPARQL Endpoints}{
    \begin{multicols}{2}
      \begin{center}
        \HRULE
        \textit{Integrate sampling as SQL did with the TABLESAMPLE clause.}
        \HRULE
      \end{center}
      
      Sample-based approximate query processing (S-AQP) has many
      important use cases for RDF such as computing
      \begin{itemize}
      \item large-scale statistics,
      \item knowledge graph embeddings,
      \item join orders,
      \item approximate aggregations,
      \item summaries,
      \item exploratory queries.
      \end{itemize}
      \ \\
      By confining query execution to samples of large datasets,
      S-AQP drastically
      \begin{itemize}
      \item reduces execution time, 
      \item delivering approximate results
      \item with error estimates.
      \end{itemize}
    \end{multicols}%
  }
  \draw[line width=0.2cm] (blocktitle.north east) -- (blockbody.south east);%
  %
  \column{0.26}
  \block{Ad-hoc sampling}{
    \begin{center}
      \HRULE
      \textit{is not great!}
      \HRULE
      
      \uppercase{May time out on public SPARQL endpoints\ldots}
      \HRULE
    \end{center}
    
    \texttt{SELECT * \{?s ?p ?o\} OFFSET r LIMIT 1}, where $r$ is a
    random number between 0 and the dataset size (0 $< r <$ 12B).\\

    %% Well-known engines such as
    %% Stardog\footnote{\url{https://docs.stardog.com/query-stardog/sampling-service\#sampling-service}},
    %% or
    %% Virtuoso\footnote{https://docs.openlinksw.com/virtuoso/rndsalltr/},
    %% propose an ad-hoc sampling API, but with no guarantee on complexity.\\
  }
\end{columns}

\begin{columns} 
  \column{0.5}
  \block{Random walks}{WanderJoin}
  \column{0.5}
  \block{In Apache Jena}{
    Modifying balanced trees\\
    Implementing cardinality estimation
  }
\end{columns}


\begin{columns} % See Section 4.4
  \column{0.33} % See Section 4.4
  \block{Conclusion}{
    Rather simple
  }
  \column{0.33}
  \block{Perspective}{
    More operators
  }
  \column{0.33}
  \block{}{
    \includegraphics[width=0.28\textwidth]{images/qr-code.png}
    \begin{center}
      \url{https://github.com/gdd-nantes/raw-jena}
    \end{center}
  }
\end{columns}

\end{document}
